\documentclass[]{article}
\usepackage{lmodern}
\usepackage{amssymb,amsmath}
\usepackage{ifxetex,ifluatex}
\usepackage{fixltx2e} % provides \textsubscript
\ifnum 0\ifxetex 1\fi\ifluatex 1\fi=0 % if pdftex
  \usepackage[T1]{fontenc}
  \usepackage[utf8]{inputenc}
\else % if luatex or xelatex
  \ifxetex
    \usepackage{mathspec}
  \else
    \usepackage{fontspec}
  \fi
  \defaultfontfeatures{Ligatures=TeX,Scale=MatchLowercase}
\fi
% use upquote if available, for straight quotes in verbatim environments
\IfFileExists{upquote.sty}{\usepackage{upquote}}{}
% use microtype if available
\IfFileExists{microtype.sty}{%
\usepackage{microtype}
\UseMicrotypeSet[protrusion]{basicmath} % disable protrusion for tt fonts
}{}
\usepackage[margin=1in]{geometry}
\usepackage{hyperref}
\hypersetup{unicode=true,
            pdftitle={Red List Index of Ecosystems},
            pdfauthor={Jessica A. Rowland},
            pdfborder={0 0 0},
            breaklinks=true}
\urlstyle{same}  % don't use monospace font for urls
\usepackage{color}
\usepackage{fancyvrb}
\newcommand{\VerbBar}{|}
\newcommand{\VERB}{\Verb[commandchars=\\\{\}]}
\DefineVerbatimEnvironment{Highlighting}{Verbatim}{commandchars=\\\{\}}
% Add ',fontsize=\small' for more characters per line
\usepackage{framed}
\definecolor{shadecolor}{RGB}{248,248,248}
\newenvironment{Shaded}{\begin{snugshade}}{\end{snugshade}}
\newcommand{\KeywordTok}[1]{\textcolor[rgb]{0.13,0.29,0.53}{\textbf{#1}}}
\newcommand{\DataTypeTok}[1]{\textcolor[rgb]{0.13,0.29,0.53}{#1}}
\newcommand{\DecValTok}[1]{\textcolor[rgb]{0.00,0.00,0.81}{#1}}
\newcommand{\BaseNTok}[1]{\textcolor[rgb]{0.00,0.00,0.81}{#1}}
\newcommand{\FloatTok}[1]{\textcolor[rgb]{0.00,0.00,0.81}{#1}}
\newcommand{\ConstantTok}[1]{\textcolor[rgb]{0.00,0.00,0.00}{#1}}
\newcommand{\CharTok}[1]{\textcolor[rgb]{0.31,0.60,0.02}{#1}}
\newcommand{\SpecialCharTok}[1]{\textcolor[rgb]{0.00,0.00,0.00}{#1}}
\newcommand{\StringTok}[1]{\textcolor[rgb]{0.31,0.60,0.02}{#1}}
\newcommand{\VerbatimStringTok}[1]{\textcolor[rgb]{0.31,0.60,0.02}{#1}}
\newcommand{\SpecialStringTok}[1]{\textcolor[rgb]{0.31,0.60,0.02}{#1}}
\newcommand{\ImportTok}[1]{#1}
\newcommand{\CommentTok}[1]{\textcolor[rgb]{0.56,0.35,0.01}{\textit{#1}}}
\newcommand{\DocumentationTok}[1]{\textcolor[rgb]{0.56,0.35,0.01}{\textbf{\textit{#1}}}}
\newcommand{\AnnotationTok}[1]{\textcolor[rgb]{0.56,0.35,0.01}{\textbf{\textit{#1}}}}
\newcommand{\CommentVarTok}[1]{\textcolor[rgb]{0.56,0.35,0.01}{\textbf{\textit{#1}}}}
\newcommand{\OtherTok}[1]{\textcolor[rgb]{0.56,0.35,0.01}{#1}}
\newcommand{\FunctionTok}[1]{\textcolor[rgb]{0.00,0.00,0.00}{#1}}
\newcommand{\VariableTok}[1]{\textcolor[rgb]{0.00,0.00,0.00}{#1}}
\newcommand{\ControlFlowTok}[1]{\textcolor[rgb]{0.13,0.29,0.53}{\textbf{#1}}}
\newcommand{\OperatorTok}[1]{\textcolor[rgb]{0.81,0.36,0.00}{\textbf{#1}}}
\newcommand{\BuiltInTok}[1]{#1}
\newcommand{\ExtensionTok}[1]{#1}
\newcommand{\PreprocessorTok}[1]{\textcolor[rgb]{0.56,0.35,0.01}{\textit{#1}}}
\newcommand{\AttributeTok}[1]{\textcolor[rgb]{0.77,0.63,0.00}{#1}}
\newcommand{\RegionMarkerTok}[1]{#1}
\newcommand{\InformationTok}[1]{\textcolor[rgb]{0.56,0.35,0.01}{\textbf{\textit{#1}}}}
\newcommand{\WarningTok}[1]{\textcolor[rgb]{0.56,0.35,0.01}{\textbf{\textit{#1}}}}
\newcommand{\AlertTok}[1]{\textcolor[rgb]{0.94,0.16,0.16}{#1}}
\newcommand{\ErrorTok}[1]{\textcolor[rgb]{0.64,0.00,0.00}{\textbf{#1}}}
\newcommand{\NormalTok}[1]{#1}
\usepackage{graphicx,grffile}
\makeatletter
\def\maxwidth{\ifdim\Gin@nat@width>\linewidth\linewidth\else\Gin@nat@width\fi}
\def\maxheight{\ifdim\Gin@nat@height>\textheight\textheight\else\Gin@nat@height\fi}
\makeatother
% Scale images if necessary, so that they will not overflow the page
% margins by default, and it is still possible to overwrite the defaults
% using explicit options in \includegraphics[width, height, ...]{}
\setkeys{Gin}{width=\maxwidth,height=\maxheight,keepaspectratio}
\IfFileExists{parskip.sty}{%
\usepackage{parskip}
}{% else
\setlength{\parindent}{0pt}
\setlength{\parskip}{6pt plus 2pt minus 1pt}
}
\setlength{\emergencystretch}{3em}  % prevent overfull lines
\providecommand{\tightlist}{%
  \setlength{\itemsep}{0pt}\setlength{\parskip}{0pt}}
\setcounter{secnumdepth}{0}
% Redefines (sub)paragraphs to behave more like sections
\ifx\paragraph\undefined\else
\let\oldparagraph\paragraph
\renewcommand{\paragraph}[1]{\oldparagraph{#1}\mbox{}}
\fi
\ifx\subparagraph\undefined\else
\let\oldsubparagraph\subparagraph
\renewcommand{\subparagraph}[1]{\oldsubparagraph{#1}\mbox{}}
\fi

%%% Use protect on footnotes to avoid problems with footnotes in titles
\let\rmarkdownfootnote\footnote%
\def\footnote{\protect\rmarkdownfootnote}

%%% Change title format to be more compact
\usepackage{titling}

% Create subtitle command for use in maketitle
\providecommand{\subtitle}[1]{
  \posttitle{
    \begin{center}\large#1\end{center}
    }
}

\setlength{\droptitle}{-2em}

  \title{Red List Index of Ecosystems}
    \pretitle{\vspace{\droptitle}\centering\huge}
  \posttitle{\par}
    \author{Jessica A. Rowland}
    \preauthor{\centering\large\emph}
  \postauthor{\par}
    \date{}
    \predate{}\postdate{}
  

\begin{document}
\maketitle

\subsection{Overview}\label{overview}

Note that the \texttt{echo\ =\ FALSE} parameter was added to the code
chunk to prevent printing of the R code that generated the plot.

\subsubsection{Calculate the RLIe}\label{calculate-the-rlie}

The function `danger' orders the Red List of Ecosystems risk categories
from lowest to highest risk.

\begin{Shaded}
\begin{Highlighting}[]
\NormalTok{danger <-}\StringTok{ }\ControlFlowTok{function}\NormalTok{(x)\{}
\NormalTok{  position =}\StringTok{ }\DecValTok{1}
\NormalTok{  dangerzone <-}\StringTok{ }\KeywordTok{c}\NormalTok{(}\StringTok{"NE"}\NormalTok{, }\StringTok{"DD"}\NormalTok{, }\StringTok{"LC"}\NormalTok{, }\StringTok{"NT"}\NormalTok{, }\StringTok{"VU"}\NormalTok{, }\StringTok{"EN"}\NormalTok{, }\StringTok{"CR"}\NormalTok{, }\StringTok{"CO"}\NormalTok{)}
  \ControlFlowTok{for}\NormalTok{(i }\ControlFlowTok{in} \DecValTok{1}\OperatorTok{:}\KeywordTok{length}\NormalTok{(dangerzone))\{}
    \ControlFlowTok{if}\NormalTok{(x }\OperatorTok{==}\StringTok{ }\NormalTok{dangerzone[i]) \{}
\NormalTok{      position =}\StringTok{ }\NormalTok{i}
\NormalTok{    \}}
\NormalTok{  \}}
    \KeywordTok{return}\NormalTok{(position)}
\NormalTok{\}}
\end{Highlighting}
\end{Shaded}

The function `maxcategory' uses the category ranks defined by the
function `danger'. If the risk categories for each sub-criteria are
listed in separate column, the function `maxcategory' selects the
highest risk category across the columns (i.e.~subcritera) for each
criteria

\begin{Shaded}
\begin{Highlighting}[]
\NormalTok{maxcategory <-}\StringTok{ }\ControlFlowTok{function}\NormalTok{ (x) \{}
\NormalTok{  value =}\StringTok{ }\DecValTok{0}
\NormalTok{  position =}\StringTok{ }\DecValTok{0}
\NormalTok{  highestvalue =}\StringTok{ }\OtherTok{NULL}
  \ControlFlowTok{for}\NormalTok{(i }\ControlFlowTok{in} \DecValTok{1}\OperatorTok{:}\KeywordTok{length}\NormalTok{(x))\{}
  \ControlFlowTok{if}\NormalTok{ (}\KeywordTok{danger}\NormalTok{(x[i]) }\OperatorTok{>}\StringTok{ }\NormalTok{value)}
\NormalTok{  \{}
\NormalTok{    value =}\StringTok{ }\KeywordTok{danger}\NormalTok{(x[i])}
\NormalTok{    position =}\StringTok{ }\NormalTok{i}
\NormalTok{    highestvalue =}\StringTok{ }\NormalTok{x[i]}
\NormalTok{  \}}
\NormalTok{  \}}
\NormalTok{  category_list <-}\StringTok{ }\KeywordTok{c}\NormalTok{(highestvalue, position)}
  \KeywordTok{return}\NormalTok{(category_list)}
\NormalTok{\}}
\end{Highlighting}
\end{Shaded}

\section{For example\ldots{}}\label{for-example}

\begin{Shaded}
\begin{Highlighting}[]
\CommentTok{# Example using dataset}
\end{Highlighting}
\end{Shaded}

\subsection{Assign ordinal values to risk
categories}\label{assign-ordinal-values-to-risk-categories}

The function `calcWeights' allocates each Red List of Ecosystems risk
category an ordinal rank from 0 (Least Concern) to 5 (Collapsed)

@param eco.risk.data = dataframe\\
@param RLE.criteria = column name of criterion of interest

\begin{Shaded}
\begin{Highlighting}[]
\CommentTok{# 1) Function to calculate category weights}
\NormalTok{calcWeights <-}\StringTok{ }\ControlFlowTok{function}\NormalTok{(eco.risk.data, RLE.criteria) \{}
  \CommentTok{# Remove NA values (where values aren't true NAs)}
\NormalTok{  eco.risk.data <-}\StringTok{ }\KeywordTok{filter}\NormalTok{(eco.risk.data, .data[[RLE.criteria]] }\OperatorTok{!=}\StringTok{ "NA"}\NormalTok{)}
  
  \CommentTok{# Calculate numerical weights for each ecosystem based on risk category}
\NormalTok{  weight.data <-}\StringTok{ }\NormalTok{dplyr}\OperatorTok{::}\KeywordTok{mutate}\NormalTok{(eco.risk.data, }
                               \DataTypeTok{category.weights =} \KeywordTok{case_when}\NormalTok{(.data[[RLE.criteria]] }\OperatorTok{==}\StringTok{ "CO"} \OperatorTok{~}\StringTok{ }\DecValTok{5}\NormalTok{,}
\NormalTok{                                                            .data[[RLE.criteria]] }\OperatorTok{==}\StringTok{ "CR"} \OperatorTok{~}\StringTok{ }\DecValTok{4}\NormalTok{, }
\NormalTok{                                                            .data[[RLE.criteria]] }\OperatorTok{==}\StringTok{ "EN"} \OperatorTok{~}\StringTok{ }\DecValTok{3}\NormalTok{, }
\NormalTok{                                                            .data[[RLE.criteria]] }\OperatorTok{==}\StringTok{ "VU"} \OperatorTok{~}\StringTok{ }\DecValTok{2}\NormalTok{, }
\NormalTok{                                                            .data[[RLE.criteria]] }\OperatorTok{==}\StringTok{ "NT"} \OperatorTok{~}\StringTok{ }\DecValTok{1}\NormalTok{,}
\NormalTok{                                                            .data[[RLE.criteria]] }\OperatorTok{==}\StringTok{ "LC"} \OperatorTok{~}\StringTok{ }\DecValTok{0}\NormalTok{))}
\NormalTok{\}}
\end{Highlighting}
\end{Shaded}

\subsection{Calculate the index}\label{calculate-the-index}

The function `calcRLIe' selects the column in a dataframe listing the
risk categories and allocates the ordinal ranks allocated by the
function `calcWeights'. These ordinal ranks are then used to calculate
the Red List Index for Ecosystems (RLIe) and percentiles calturing the
middle 95\% of the data.

@param eco.risk.data = dataframe\\
@param RLE.criteria = column name of criterion of interest\\
@param group = the factor (optional) you want to group the index by.
Where not specified, an RLIe will be calculated based on all ecosystems
(output = single score)\\
@param group2 = the second factor (optional) you want to group the index
by

\begin{Shaded}
\begin{Highlighting}[]
\NormalTok{calcRLIe <-}\StringTok{ }\ControlFlowTok{function}\NormalTok{(eco.risk.data, RLE.criteria, group, group2)\{}
  
  \CommentTok{# Filter out rows with NE and DD}
\NormalTok{  filter.data <-}\StringTok{ }\NormalTok{dplyr}\OperatorTok{::}\KeywordTok{filter}\NormalTok{(eco.risk.data, .data[[RLE.criteria]] }\OperatorTok{!=}\StringTok{ "NE"} \OperatorTok{&}\StringTok{ }\NormalTok{.data[[RLE.criteria]] }\OperatorTok{!=}\StringTok{ "DD"}\NormalTok{)}
  
  \CommentTok{# Calculate numerical weights for each ecosystem based on risk category}
\NormalTok{  weight.data <-}\StringTok{ }\KeywordTok{calcWeights}\NormalTok{(filter.data, RLE.criteria)}
\NormalTok{  weight.data <-}\StringTok{ }\KeywordTok{drop_na}\NormalTok{(weight.data, .data[[RLE.criteria]])}
  
  \CommentTok{# Calculate overall RLIe score for each year}
  \ControlFlowTok{if}\NormalTok{ (}\KeywordTok{missing}\NormalTok{(group)) \{}
\NormalTok{    values <-}\StringTok{ }\KeywordTok{group_by}\NormalTok{(weight.data)}
    
    \CommentTok{# Calculate scores for each group level in each year if specfied}
\NormalTok{  \} }\ControlFlowTok{else}\NormalTok{ \{}
    \ControlFlowTok{if}\NormalTok{ (}\KeywordTok{missing}\NormalTok{(group2)) \{}
\NormalTok{      values <-}\StringTok{ }\KeywordTok{group_by}\NormalTok{(weight.data, }\DataTypeTok{group =}\NormalTok{ .data[[group]])}
      \CommentTok{# Calculate scores for each level within two nested groupings in each year if specified}
\NormalTok{    \}  }\ControlFlowTok{else}\NormalTok{ \{}
\NormalTok{      values <-}\StringTok{ }\KeywordTok{group_by}\NormalTok{(weight.data, }\DataTypeTok{group =}\NormalTok{ .data[[group]],}
                         \DataTypeTok{group2 =}\NormalTok{ .data[[group2]])}
\NormalTok{    \}}
\NormalTok{  \}}
  \CommentTok{# Sum category weights for each group, calculate number of ecosystems per group}
\NormalTok{  summed.weights <-}\StringTok{ }\KeywordTok{summarise}\NormalTok{(values, }\DataTypeTok{total.weight =} \KeywordTok{sum}\NormalTok{(category.weights), }\DataTypeTok{total.count =} \KeywordTok{n}\NormalTok{(), }
                              \DataTypeTok{upper.95.quant =} \DecValTok{1} \OperatorTok{-}\StringTok{ }\KeywordTok{quantile}\NormalTok{(category.weights, }\DataTypeTok{probs =} \FloatTok{0.025}\NormalTok{) }\OperatorTok{/}\StringTok{ }\DecValTok{5}\NormalTok{, }
                              \DataTypeTok{lower.95.quant =} \DecValTok{1} \OperatorTok{-}\StringTok{ }\KeywordTok{quantile}\NormalTok{(category.weights, }\DataTypeTok{probs =} \FloatTok{0.975}\NormalTok{) }\OperatorTok{/}\StringTok{ }\DecValTok{5}\NormalTok{,}
                              \DataTypeTok{max =} \DecValTok{1} \OperatorTok{-}\StringTok{ }\KeywordTok{min}\NormalTok{(category.weights) }\OperatorTok{/}\StringTok{ }\DecValTok{5}\NormalTok{,}
                              \DataTypeTok{min =} \DecValTok{1} \OperatorTok{-}\StringTok{ }\KeywordTok{max}\NormalTok{(category.weights) }\OperatorTok{/}\StringTok{ }\DecValTok{5}\NormalTok{)}
  
  \CommentTok{# Calculate RLIe scores for each group, rounded to 3 decimal places}
\NormalTok{  index.scores <-}\StringTok{ }\KeywordTok{mutate}\NormalTok{(summed.weights, }\DataTypeTok{RLIe =} \DecValTok{1} \OperatorTok{-}\StringTok{ }\NormalTok{(total.weight}\OperatorTok{/}\NormalTok{(total.count }\OperatorTok{*}\StringTok{ }\DecValTok{5}\NormalTok{)),}
                         \DataTypeTok{Criteria =}\NormalTok{ RLE.criteria)}
  
  \CommentTok{# output <- dplyr::select(RLIe.scores, "Group" = group, "Total.count" = total.count, "RLIe" = RLIe)}
  \KeywordTok{return}\NormalTok{(index.scores)}
\NormalTok{\}}
\CommentTok{# rlie.output <- calcRLIe(data, RLE.criteria = "criterion.A", group = "ecosystem.group")}
\end{Highlighting}
\end{Shaded}


\end{document}
